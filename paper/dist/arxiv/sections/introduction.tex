% Introduction Section

\section{Introduction}
\label{sec:intro}

Galaxy rotation curves remain a stress test for any theory that seeks to explain the flattened outer profiles of spiral galaxies without an ad hoc halo tuned to each system.  While $\Lambda$CDM fits typically invoke two halo parameters per galaxy and MOND introduces a low-acceleration law, direct predictive comparisons across paradigms rarely enforce matched parameter budgets.  As a result, literature claims often contrast a predictive model with a descriptive one, obscuring whether a purported advantage comes from new physics or from extra degrees of freedom.

We frame the problem explicitly in terms of the per-galaxy parameter budget $k$.  Predictive settings fix $k=0$---all parameters are learned on TRAIN and frozen on blind TEST---whereas descriptive settings allow $k \geq 2$ per galaxy.  Our baselines therefore include three k=0 models (RFT v2, global NFW, MOND) plus a k=2 reference (per-galaxy NFW fits) that represents the descriptive ceiling but is not a fair competitor.  Table~\ref{tab:param_budget} summarises these budgets and the degrees of freedom allocated to each solver.

RFT v2 extends the geometry-only kernel developed in earlier RFT campaigns with a frozen acceleration-gated tail and an identity convolution kernel, yielding six global parameters and zero per-galaxy tuning.  On the blind TEST cohort (\testCohortSize galaxies) this predictive configuration passes the 20\% RMS criterion in \rftPassTwentyFrac galaxies (\rftPassTwentyPct), compared with \nfwPassTwentyFrac (\nfwPassTwentyPct) for the global NFW halo and \mondPassTwentyFrac (\mondPassTwentyPct) for canonical MOND.  The paired McNemar test indicates that the RFT vs NFW difference is suggestive but \emph{not} statistically significant ($p=\mcnemarPvsNFW$), whereas the RFT vs MOND gap is highly significant ($p=\mcnemarPvsMOND$).  These results therefore motivate honest framing: RFT is competitive with a fair NFW baseline and significantly ahead of MOND, but it does not yet deliver a decisive win over $\Lambda$CDM.

This paper is scoped strictly to the galaxy rotation campaign.  The L-Series cluster lensing program reported a negative predictive result in its final audit, so no claims about lensing or cosmological-scale performance are made here; those tracks remain separate until they achieve the same level of rigor.  Our goal is to document what is and is not established by the SPARC-99 study and to freeze the accompanying data/figure assets for peer review.

Our contributions are: (i) a transparent predictive comparison between RFT v2 and k=0 baselines with matched parameter budgets; (ii) a fairness pack that enumerates head-to-head wins, Wilson confidence intervals, and LSB/HSB behavior; (iii) robustness evidence via $\pm$10\% parameter perturbations, ablations, and representative rotation-curve galleries; and (iv) complete reproducibility assets (RUNME, CI gates, and hash-locked numbers) that allow anyone to regenerate the figures and tables in Sections~\ref{sec:results}--\ref{sec:discussion}.  The remainder of the paper proceeds as follows: Section~\ref{sec:methods} details the dataset, metrics, and computation pipeline; Section~\ref{sec:results} presents the fairness, robustness, and gallery analyses; Section~\ref{sec:discussion} interprets the findings and limitations; and Section~\ref{sec:conclusion} provides conclusions and future work.
