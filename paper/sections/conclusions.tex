% Conclusions Section

\section{Conclusions}
\label{sec:conclusion}

\begin{itemize}
    \item \textbf{Predictive result}: RFT v2 achieves \rftPassTwentyPct\ pass@20\% on blind TEST using six global parameters and $k=0$.  It is competitive with the global NFW halo (McNemar $p=\mcnemarPvsNFW$) and significantly ahead of MOND ($p=\mcnemarPvsMOND$).
    \item \textbf{Mechanism validated}: The acceleration-gated tail is supported by LSB dominance (Figure~\ref{fig:lsb}), $\pm$10\% stability sweeps (Figure~\ref{fig:stability}), ablations (Figure~\ref{fig:ablations}), and rotation-curve galleries (Figure~\ref{fig:gallery}).
    \item \textbf{Limitations}: Only six discordant galaxies separate RFT and NFW, \rftFailCount\ galaxies still fail the 20\% criterion, and the identity kernel leaves bulge-dominated systems under-fit.  Larger cohorts and additional physics are required for decisive claims.
    \item \textbf{Future work}: (1) expand the blind cohort to improve statistical power; (2) run joint RC+lensing tests so that the galaxy track and the L-Series cluster track share constraints; (3) explore per-galaxy or descriptor-driven RFT variants as a separate, explicitly descriptive question; and (4) continue pre-registered follow-ups so that any new tuning cycles are audit-ready.
\end{itemize}
